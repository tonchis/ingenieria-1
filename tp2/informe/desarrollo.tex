‌\section{Introducción}
Con este diseño se busca un sistema de venta de telefonía celular que permita tener un alto grado de confiabilidad tanto para el vendedor como para el cliente, que mantenga interconectadas a las distintas partes de la empresa (marketing - ventas - facturación - stock) en todo momento, que automatice algunos procesos internos de la empresa y que agilice y mejore el rendimiento de ventas.\\
\indent Para presentar esta propuesta se incluyen dos alternativas de Diagramas de contexto, donde se visualizan las interacciones entre los distintos agentes y el sistema con un listado del detalle de la misma. Por otro lado, se muestra un Diagrama de Objetivos el cual propone los distintos hitos y metas que deberán cumplirse para lograr satisfacer el objetivo global del diseño. Para cada rama importante de este árbol de objetivos se presenta un detalle de escenarios en donde se ejemplifican, de manera coloquial, distintos sucesos de la vida real acordes al sistema implementado.\\

\section{Modelo Conceptual}

\clearpage

\section{Casos de Uso}

\begin{tabular}{ | p{7cm} | p{7cm} | }
  \hline
  \multicolumn{2} {|l|} {Caso de Uso: Autenticándose} \\
  \multicolumn{2} {|l|} {Actores: Vendedor, Encargado de Stock, Facturación, Marketing} \\
  \multicolumn{2} {|l|} {Post: El usuario está autenticado en el sistema} \\
  \hline
  Curso Normal & Curso Alternativo\\
  \hline
  1. El usuario ingresa a la interfaz web del sistema. & \\
  2. El usuario ingresa sus credenciales. & \\
  3. El sistema comprueba la validez de las credenciales y las acepta. & 3. El sistema comprueba la validez de las credenciales y las rechaza\\
  4. El usuario queda autenticado en el sistema. & 4. El usuario es redirigido al formulario para reingresar sus credenciales.\\
  \hline
\end{tabular}

\clearpage

\section{Diagramas de actividad}

\clearpage
