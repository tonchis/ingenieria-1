‌\section{Introducción}
Con este diseño se busca un sistema de venta de telefonía celular que permita tener un alto grado de confiabilidad tanto para el vendedor como para el cliente, que mantenga interconectadas a las distintas partes de la empresa (marketing - ventas - facturación - stock) en todo momento, que automatice algunos procesos internos de la empresa y que agilice y mejore el rendimiento de ventas.\\
\indent Para presentar esta propuesta se incluyen dos alternativas de Diagramas de contexto, donde se visualizan las interacciones entre los distintos agentes y el sistema con un listado del detalle de la misma. Por otro lado, se muestra un Diagrama de Objetivos el cual propone los distintos hitos y metas que deberán cumplirse para lograr satisfacer el objetivo global del diseño. Para cada rama importante de este árbol de objetivos se presenta un detalle de escenarios en donde se ejemplifican, de manera coloquial, distintos sucesos de la vida real acordes al sistema implementado.\\

\section{Modelo Conceptual}

\clearpage

\section{Casos de Uso}

\subsection{Casos de Uso MVP}

\subsubsection{Descripción de Casos de Uso}

\begin{tabular}{ | p{7cm} | p{7cm} | }
  \hline
  \multicolumn{2} {|l|} {Caso de Uso: Autenticándose.} \\
  \multicolumn{2} {|l|} {Actores: Vendedor, Encargado de Stock, Facturación, Marketing.} \\
  \multicolumn{2} {|l|} {Post: El usuario está autenticado en el sistema.} \\
  \hline
  Curso Normal & Curso Alternativo\\
  \hline
  1. El usuario ingresa a la interfaz web del sistema. & \\
  2. El usuario ingresa sus credenciales. & \\
  3. El sistema comprueba la validez de las credenciales y las acepta. & 3. El sistema comprueba la validez de las credenciales y las rechaza\\
  4. El usuario queda autenticado en el sistema. & 4. El usuario es redirigido al formulario para reingresar sus credenciales.\\
  \hline
\end{tabular}

\vspace{1cm}

\begin{tabular}{ | p{14cm} | }
  \hline
  Caso de Uso: Actualizando datos via terminal. \\
  Actores: Vendedor. \\
  Pre: El usuario está autenticado en el sistema. \\
  Post: El vendedor tiene actualizados en su dispositivo móvil promociones y datos de clientes. \\
  \hline
  Curso Normal\\
  \hline
  1. El vendedor enchufa su dispositivo móvil en una terminal del sistema ubicada en la oficina. \\
  2. El sistema actualiza automáticamente datos de promociones y clientes. \\
  \hline
\end{tabular}

\vspace{1cm}

\begin{tabular}{ | p{14cm} | }
  \hline
  Caso de Uso: Consultando datos via Internet. \\
  Actores: Vendedor. \\
  Pre: El usuario está autenticado en el sistema. \\
  Post: El vendedor recibe en su dispositivo móvil promociones y datos de clientes. \\
  \hline
  Curso Normal\\
  \hline
  1. El vendedor ingresa nombre del cliente que está visitando en su dispositivo móvil. \\
  2. El sistema busca los datos relacionados al cliente y promociones disponibles para él y los envía al dispositivo móvil del vendedor. \\
  \hline
\end{tabular}

\vspace{1cm}

\begin{tabular}{ | p{14cm} | }
  \hline
  Caso de Uso: Avisando stock bajo. \\
  Actores: Encargado de Stock. \\
  Pre: El usuario está autenticado en el sistema y l stock de alguno de los equipos asociado a promociones vigentes está pronto a agotarse. \\
  Post: El encargado de stock es notificado sobre el equipo pronto a agotarse. \\
  \hline
  Curso Normal\\
  \hline
  1. El sistema marca como reservados los equipos correspondientes con la promoción vendida.\\
  2. El sistema notifica al encargado de stock sobre los equipos reservados.\\
  \hline
\end{tabular}

\vspace{1cm}

\begin{tabular}{ | p{7cm} | p{7cm} | }
  \hline
  \multicolumn{2} {|l|} {Caso de Uso: Verificando venta.} \\
  \multicolumn{2} {|l|} {Actores: Facturación.} \\
  \multicolumn{2} {|l|} {Pre: El usuario está autenticado con el sistema y se registró una venta con stock disponible.} \\
  \multicolumn{2} {|l|} {Post: La venta se confirma y pasa al sistema de facturación.} \\
  \hline
  Curso Normal & Curso Alternativo\\
  \hline
  1. Facturación verifica la situación del cliente y aprueba la venta. & 1. Facturación verifica la situación del cliente y rechaza la venta. \\
  2. Se pasan los datos al sistema de facturación & 2. Se notifica al vendedor que la venta fue rechazada. \\
  \hline
\end{tabular}

\vspace{1cm}

\begin{tabular}{ | p{7cm} | p{7cm} | }
  \hline
   \multicolumn{2} {| p{10cm} |} {Caso de Uso: Registrando venta.} \\
   \multicolumn{2} {| p{10cm} |} {Actores: Vendedor, Encargado de Stock, Facturación.} \\
   \multicolumn{2} {| p{10cm} |} {Pre: El usuario está autenticado con el sistema y el vendedor tiene datos de clientes y promociones actualizados.} \\
   \multicolumn{2} {| p{10cm} |} {Post: Se registra una venta en el sistema.} \\
  \hline
  Curso Normal & Curso Alternativo\\
  \hline
  1. Se extiende con Consultando datos via Internet. & \\
  2. El vendedor decide junto al cliente qué promoción adquirir. & \\
  3. El vendedor ingresa elige en su dispositivo móvil la promoción que su cliente desea. & \\
  4. El sistema verifica que haya suficientes equipos en stock para satisfacer la promoción. & 4. El sistema verifica que los equipos en stock no alcanzan para satisfacer la promoción\\
  5. El sistema marca como reservados los equipos correspondientes con la promoción vendida. & 5. Se notifica al vendedor que la promoción no puede ser vendida. \\
  6. El sistema notifica al encargado de stock sobre los equipos reservados. & \\ 
  7. Se extiende con Avisando stock bajo. & \\
  8. Usa Verificando venta. & \\
  \hline
\end{tabular}

\vspace{1cm}

\begin{tabular}{ | p{14cm} | }
  \hline
  Caso de Uso: Ingresando equipos nuevos. \\
  Actores: Encargado de Stock. \\
  Pre: El usuario está autenticado con el sistema y nuevos equipos fueron ingresados al almacén. \\
  Post: Los nuevos equipos quedan dados de alta en el sistema. \\
  \hline
  Curso Normal\\
  \hline
  1. El usuario actualiza la cantidad de equipos correspondientes a los ingresados recientemente. \\
  \hline
\end{tabular}

\vspace{1cm}

\begin{tabular}{ | p{14cm} | }
  \hline
  Caso de Uso: Administrando promociones. \\
  Actores: Marketing. \\
  Pre: El usuario está autenticado con el sistema. \\
  Post: Se actualizan las promociones disponibles. \\
  \hline
  Curso Normal\\
  \hline
  1. El departamento de Marketing realiza un estudio de mercado para decidir qué promociones ofrecer según tipo de cliente. \\
  2. El usuario ingresado en el sistema crea, edita o borra una promoción. \\
  \hline
\end{tabular}

\subsection{Casos de Uso}

\subsubsection{Descripción de Casos de Uso}

\begin{tabular}{ | p{7cm} | p{7cm} | }
  \hline
  \multicolumn{2} {|l|} {Caso de Uso: Autenticándose.} \\
  \multicolumn{2} {|l|} {Actores: Vendedor, Encargado de Stock, Facturación, Marketing.} \\
  \multicolumn{2} {|l|} {Post: El usuario está autenticado en el sistema.} \\
  \hline
  Curso Normal & Curso Alternativo\\
  \hline
  1. El usuario ingresa a la interfaz web del sistema. & \\
  2. El usuario ingresa sus credenciales. & \\
  3. El sistema comprueba la validez de las credenciales y las acepta. & 3. El sistema comprueba la validez de las credenciales y las rechaza\\
  4. El usuario queda autenticado en el sistema. & 4. El usuario es redirigido al formulario para reingresar sus credenciales.\\
  \hline
\end{tabular}

\vspace{1cm}

\begin{tabular}{ | p{14cm} | }
  \hline
  Caso de Uso: Actualizando datos via terminal. \\
  Actores: Vendedor. \\
  Pre: El usuario está autenticado en el sistema. \\
  Post: El vendedor tiene actualizados en su dispositivo móvil promociones y datos de clientes. \\
  \hline
  Curso Normal\\
  \hline
  1. El vendedor enchufa su dispositivo móvil en una terminal del sistema ubicada en la oficina. \\
  2. El sistema actualiza automáticamente datos de promociones y clientes. \\
  \hline
\end{tabular}

\vspace{1cm}

\begin{tabular}{ | p{14cm} | }
  \hline
  Caso de Uso: Consultando datos via Internet. \\
  Actores: Vendedor. \\
  Pre: El usuario está autenticado en el sistema. \\
  Post: El vendedor recibe en su dispositivo móvil promociones y datos de clientes. \\
  \hline
  Curso Normal\\
  \hline
  1. El vendedor ingresa nombre del cliente que está visitando en su dispositivo móvil. \\
  2. El sistema busca los datos relacionados al cliente y promociones disponibles para él y los envía al dispositivo móvil del vendedor. \\
  \hline
\end{tabular}

\vspace{1cm}

\begin{tabular}{ | p{14cm} | }
  \hline
  Caso de Uso: Avisando stock bajo. \\
  Actores: Encargado de Stock. \\
  Pre: El usuario está autenticado en el sistema y l stock de alguno de los equipos asociado a promociones vigentes está pronto a agotarse. \\
  Post: El encargado de stock es notificado sobre el equipo pronto a agotarse. \\
  \hline
  Curso Normal\\
  \hline
  1. El sistema marca como reservados los equipos correspondientes con la promoción vendida.\\
  2. El sistema notifica al encargado de stock sobre los equipos reservados.\\
  \hline
\end{tabular}

\vspace{1cm}

\begin{tabular}{ | p{7cm} | p{7cm} | }
  \hline
  \multicolumn{2} {|l|} {Caso de Uso: Verificando venta.} \\
  \multicolumn{2} {|l|} {Actores: Facturación.} \\
  \multicolumn{2} {|l|} {Pre: El usuario está autenticado con el sistema y se registró una venta con stock disponible.} \\
  \multicolumn{2} {|l|} {Post: La venta se confirma y pasa al sistema de facturación.} \\
  \hline
  Curso Normal & Curso Alternativo\\
  \hline
  1. Facturación verifica la situación del cliente y aprueba la venta. & 1. Facturación verifica la situación del cliente y rechaza la venta. \\
  2. Se pasan los datos al sistema de facturación & 2. Se notifica al vendedor que la venta fue rechazada. \\
  \hline
\end{tabular}

\vspace{1cm}

\begin{tabular}{ | p{7cm} | p{7cm} | }
  \hline
   \multicolumn{2} {| p{10cm} |} {Caso de Uso: Registrando venta.} \\
   \multicolumn{2} {| p{10cm} |} {Actores: Vendedor, Encargado de Stock, Facturación.} \\
   \multicolumn{2} {| p{10cm} |} {Pre: El usuario está autenticado con el sistema y el vendedor tiene datos de clientes y promociones actualizados.} \\
   \multicolumn{2} {| p{10cm} |} {Post: Se registra una venta en el sistema.} \\
  \hline
  Curso Normal & Curso Alternativo\\
  \hline
  1. Se extiende con Consultando datos via Internet. & \\
  2. El vendedor decide junto al cliente qué promoción adquirir. & \\
  3. El vendedor ingresa elige en su dispositivo móvil la promoción que su cliente desea. & \\
  4. El sistema verifica que haya suficientes equipos en stock para satisfacer la promoción. & 4. El sistema verifica que los equipos en stock no alcanzan para satisfacer la promoción\\
  5. El sistema marca como reservados los equipos correspondientes con la promoción vendida. & 5. Se notifica al vendedor que la promoción no puede ser vendida. \\
  6. El sistema notifica al encargado de stock sobre los equipos reservados. & \\ 
  7. Se extiende con Avisando stock bajo. & \\
  8. Usa Verificando venta. & \\
  \hline
\end{tabular}

\vspace{1cm}

\begin{tabular}{ | p{14cm} | }
  \hline
  Caso de Uso: Ingresando equipos nuevos. \\
  Actores: Encargado de Stock. \\
  Pre: El usuario está autenticado con el sistema y nuevos equipos fueron ingresados al almacén. \\
  Post: Los nuevos equipos quedan dados de alta en el sistema. \\
  \hline
  Curso Normal\\
  \hline
  1. El usuario actualiza la cantidad de equipos correspondientes a los ingresados recientemente. \\
  \hline
\end{tabular}

\vspace{1cm}

\begin{tabular}{ | p{14cm} | }
  \hline
  Caso de Uso: Administrando promociones. \\
  Actores: Marketing. \\
  Pre: El usuario está autenticado con el sistema. \\
  Post: Se actualizan las promociones disponibles. \\
  \hline
  Curso Normal\\
  \hline
  1. El departamento de Marketing realiza un estudio de mercado para decidir qué promociones ofrecer según tipo de cliente. \\
  2. El usuario ingresado en el sistema crea, edita o borra una promoción. \\
  \hline
\end{tabular}

\vspace{1cm}

\begin{tabular}{ | p{14cm} | }
  \hline
  Caso de Uso: Seleccionando sitios y usuarios para monitorear. \\
  Actores: Marketing. \\
  Pre: El usuario está autenticado con el sistema. \\
  Post: \\
  \hline
  Curso Normal\\
  \hline
  \hline
\end{tabular}

\vspace{1cm}

\begin{tabular}{ | p{14cm} | }
  \hline
  Caso de Uso: Validando promoción automática de stock acumulado. \\
  Actores: Marketing. \\
  Pre: El usuario está autenticado con el sistema. \\
  Post: \\
  \hline
  Curso Normal\\
  \hline
  \hline
\end{tabular}

\vspace{1cm}

\begin{tabular}{ | p{14cm} | }
  \hline
  Caso de Uso: Validando promoción automática de mercado. \\
  Actores: Marketing. \\
  Pre: El usuario está autenticado con el sistema. \\
  Post: \\
  \hline
  Curso Normal\\
  \hline
  \hline
\end{tabular}

\vspace{1cm}

\begin{tabular}{ | p{14cm} | }
  \hline
  Caso de Uso: Monitoreando medios.\\
  Actores: Marketing. \\
  Pre: El usuario está autenticado con el sistema y se seleccionaron sitios y usuarios para monitorear. \\
  Post: \\
  \hline
  Curso Normal\\
  \hline
  \hline
\end{tabular}

\vspace{1cm}

\begin{tabular}{ | p{14cm} | }
  \hline
  Caso de Uso: Recibiendo informe de monitoreo. \\
  Actores: Marketing. \\
  Pre: El usuario está autenticado con el sistema. \\
  Post: \\
  \hline
  Curso Normal\\
  \hline
  \hline
\end{tabular}

\clearpage

\section{Diagramas de actividad}

\clearpage
