‌\section{Introducción}
Con este diseño se busca un sistema de venta de telefonía celular que permita tener un alto grado de confiabilidad tanto para el vendedor como para el cliente, que mantenga interconectadas a las distintas partes de la empresa (marketing - ventas - facturación - stock) en todo momento, que automatice algunos procesos internos de la empresa y que agilice y mejore el rendimiento de ventas.\\
\indent Para presentar esta propuesta se incluyen dos alternativas de Diagramas de contexto, donde se visualizan las interacciones entre los distintos agentes y el sistema con un listado del detalle de la misma. Por otro lado, se muestra un Diagrama de Objetivos el cual propone los distintos hitos y metas que deberán cumplirse para lograr satisfacer el objetivo global del diseño. Para cada rama importante de este árbol de objetivos se presenta un detalle de escenarios en donde se ejemplifican, de manera coloquial, distintos sucesos de la vida real acordes al sistema implementado.\\

\section{Modelo Conceptual}

\clearpage

\section{Casos de Uso}

\subsection{Casos de Uso MVP}

\subsubsection{Descripción de Casos de Uso}

\begin{tabular}{ | p{7cm} | p{7cm} | }
  \hline
  \multicolumn{2} {|l|} {Caso de Uso: Autenticándose.} \\
  \multicolumn{2} {|l|} {Actores: Vendedor, Encargado de Stock, Facturación, Marketing.} \\
  \multicolumn{2} {|l|} {Post: El usuario está autenticado en el sistema.} \\
  \hline
  Curso Normal & Curso Alternativo\\
  \hline
  1. El usuario ingresa a la interfaz web del sistema. & \\
  2. El usuario ingresa sus credenciales. & \\
  3. El sistema comprueba la validez de las credenciales y las acepta. & 3. El sistema comprueba la validez de las credenciales y las rechaza\\
  4. El usuario queda autenticado en el sistema. & 4. El usuario es redirigido al formulario para reingresar sus credenciales.\\
  \hline
\end{tabular}

\begin{tabular}{ | p{7cm} | p{7cm} | }
  \hline
  Caso de Uso: Actualizando datos. \\
  Actores: Vendedor. \\
  Pre: El usuario está autenticado en el sistema. \\
  Post: El vendedor tiene actualizados en su dispositivo móvil promociones y datos de clientes. \\
  \hline
  Curso Normal\\
  \hline
  Online o manual (sin auth)?\\
  \hline
\end{tabular}

\begin{tabular}{ | p{7cm} | p{7cm} | }
  \hline
  Caso de Uso: Reservando equipos. \\
  Actores: Encargado de Stock. \\
  Pre: Se realizó la venta de una promoción. \\
  Post: El encargado de stock es notificado y se reservan los equipos correspondientes a la promoción vendida. \\
  \hline
  Curso Normal\\
  \hline
  1. El sistema marca como reservados los equipos correspondientes con la promoción vendida.\\
  2. El sistema notifica al encargado de stock sobre los equipos reservados.\\
  \hline
\end{tabular}

\begin{tabular}{ | p{7cm} | p{7cm} | }
  \hline
  Caso de Uso: Avisando stock bajo. \\
  Actores: Encargado de Stock. \\
  Pre: El stock de alguno de los equipos asociado a promociones vigentes está pronto a agotarse. \\
  Post: El encargado de stock es notificado sobre el equipo pronto a agotarse. \\
  \hline
  Curso Normal\\
  \hline
  1. El sistema marca como reservados los equipos correspondientes con la promoción vendida.\\
  2. El sistema notifica al encargado de stock sobre los equipos reservados.\\
  \hline
\end{tabular}


\begin{tabular}{ | p{7cm} | p{7cm} | }
  \hline
  \multicolumn{2} {|l|} {Caso de Uso: Autenticándose.} \\
  \multicolumn{2} {|l|} {Actores: Vendedor, Encargado de Stock, Facturación, Marketing.} \\
  \multicolumn{2} {|l|} {Pre: El usuario está autenticado con el sistema y se registró una venta con stock disponible.} \\
  \multicolumn{2} {|l|} {Post: La venta se confirma y pasa al sistema de facturación.} \\
  \hline
  Curso Normal & Curso Alternativo\\
  \hline
  1. Facturación verifica la situación del cliente y aprueba la venta. & 1. Facturación verifica la situación del cliente y rechaza la venta. \\
  2. Se pasan los datos al sistema de facturación & 2. Se notifica al cliente que la venta fue rechazada. (?????)\\
  \hline
\end{tabular}


\subsection{Casos de Uso}

\subsubsection{Descripción de Casos de Uso}

\clearpage

\section{Diagramas de actividad}

\clearpage
