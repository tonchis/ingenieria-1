\documentclass[a4paper,10pt]{article}

\usepackage[margin=1in]{geometry} 	% Setea el margen manualmente, todos iguales.
\usepackage[spanish]{babel} 		% {Con estos dos anda
\usepackage[utf8]{inputenc} 		% todo lo que es tildes y ñ}
\usepackage{fancyhdr} 			%{Estos dos son para
\pagestyle{fancyplain} 			% el header copado}
\usepackage{color}			% Con esto puedo hacer la matufia de poner en color blanco un texto para engañar al formato
\usepackage{graphicx}	% Para insertar gráficos
\usepackage{array}			% Para usar arrays
\usepackage{hyperref}		% Para que tenga links el índice
%\usepackage{datetime}	% Para agregar automáticamente fecha/hora de compilación y otras cosas

\lhead{Ingeniería del Software II} 	% {Con esto se usa el header copado. También está \chead para
\rhead{Grupo 1} 	% el centro y comandos para el pie de página, buscar fancyhdr}
\renewcommand{\footrulewidth}{0.4pt}
\lfoot{Facultad de Ciencias Exactas y Naturales}
\rfoot{Universidad de Buenos Aires}
%\rfoot{\textit{}}
\usepackage{amsfonts}	% para simbolos de reales, naturales, etc. se usa \mathbb{•} y la letra
\usepackage{amsmath}	% para \implies
%\usepackage{algorithm}
%\usepackage{algorithmic}
\usepackage{caratula}
%%%%%%%%%%%%%%%%%%%%%%%%%%%%%%%%%%%%%
%      COMANDOS ÚTILES USADOS       %
%%%%%%%%%%%%%%%%%%%%%%%%%%%%%%%%%%%%%

% \section{title} 		Te hace un título ``importante'' en negrita, numerado. También está \subsection{title} y \subsubsection{title}.
% \begin{itemize}		Te hace viñetas.
%	\item esto es un item	Cambiar itemize por enumerate te hace una numeración.
% \end{itemize}

% \textbf{text} 		Te hace el texto en negrita (bold).
% \underline{text}		Te subraya el texto.

% \textsuperscript{text}	Te hace ``superindices'' con texto. En teoría subscript debería funcionar, pero se puede usar guion bajo entre llaves
% 				y signos peso para hacerlo como alternativa. Sino buscar.

% \begin{tabular}{cols} 	Es para hacer tablas. Se pone una c por cada columna deseada dentro de cols (si es que se desea centrada, l para justificar a 
%	a & b & c		izquierda, r a la derecha). Si se separa por espacios la tabla no tendrá líneas divisorias. Si se separa por | en lugar de 
% \end{tabular}			espacios, aparecerá una línea. Con || dos, y así. Luego para los elementos de las filas se escriben y se separan con ampersand (&).
%				Finalmente, para las líneas horizontales, se usa \hline para una linea en toda la tabla y \cline{i - j} te hace la linea desde
%				la celda i hasta la j, arrancando en 1.
%				Si en la columna se pone p(width) podés escribir un párrafo en la celda. Para hacer un enter con \\ no funciona porque te hace un
%				enter en la fila. Para eso se usa el comando \newline.
  
% \textcolor{color predefinido en palabras}{text}

%%%%%%%%%%%%%%%%%%%%%%%%%%%%%%%%%%%%%
%    FIN COMANDOS ÚTILES USADOS     %
%%%%%%%%%%%%%%%%%%%%%%%%%%%%%%%%%%%%%

\newcommand{\Gather}[1]{\begin{gather*}#1\end{gather*}}
%\newcommand{\Def}[1]{\textbf{Definición: }#1}
%\newcommand{\Prop}[1]{\textbf{Propiedad: }#1}
%\newcommand{\Teo}[1]{\textbf{Teorema: }#1}
\newcommand{\Obs}[1]{\textbf{Observación: }#1}
%\newcommand{\Amat}{A \in \mathbb{R}^{n\textnormal{x}n}}
\newcommand{\filtro}[1]{\textbf{\textit{#1}}}

\begin{document}

%%%%%%%%%%%%%%%%%%%%%%%%%%%
%			INICIO DE CARÁTULA			%
%%%%%%%%%%%%%%%%%%%%%%%%%%%

%% **************************************************************************
%
%  Package 'caratula', version 0.2 (para componer caratulas de TPs del DC).
%
%  En caso de dudas, problemas o sugerencias sobre este package escribir a
%  Nico Rosner (nrosner arroba dc.uba.ar).
%
% **************************************************************************



% ----- Informacion sobre el package para el sistema -----------------------

\NeedsTeXFormat{LaTeX2e}
\ProvidesPackage{caratula}[2003/4/13 v0.1 Para componer caratulas de TPs del DC]


% ----- Imprimir un mensajito al procesar un .tex que use este package -----

\typeout{Cargando package 'caratula' v0.2 (21/4/2003)}


% ----- Algunas variables --------------------------------------------------

\let\Materia\relax
\let\Submateria\relax
\let\Titulo\relax
\let\Subtitulo\relax
\let\Grupo\relax


% ----- Comandos para que el usuario defina las variables ------------------

\def\materia#1{\def\Materia{#1}}
\def\submateria#1{\def\Submateria{#1}}
\def\titulo#1{\def\Titulo{#1}}
\def\subtitulo#1{\def\Subtitulo{#1}}
\def\grupo#1{\def\Grupo{#1}}


% ----- Token list para los integrantes ------------------------------------

\newtoks\intlist\intlist={}


% ----- Comando para que el usuario agregue integrantes

\def\integrante#1#2#3{\intlist=\expandafter{\the\intlist
	\rule{0pt}{1.2em}#1&#2&\tt #3\\[0.2em]}}


% ----- Macro para generar la tabla de integrantes -------------------------

\def\tablaints{%
	\begin{tabular}{|l@{\hspace{4ex}}c@{\hspace{4ex}}l|}
		\hline
		\rule{0pt}{1.2em}Integrante & LU & Correo electr\'onico\\[0.2em]
		\hline
		\the\intlist
		\hline
	\end{tabular}}


% ----- Codigo para manejo de errores --------------------------------------

\def\se{\let\ifsetuperror\iftrue}
\def\ifsetuperror{%
	\let\ifsetuperror\iffalse
	\ifx\Materia\relax\se\errhelp={Te olvidaste de proveer una \materia{}.}\fi
	\ifx\Titulo\relax\se\errhelp={Te olvidaste de proveer un \titulo{}.}\fi
	\edef\mlist{\the\intlist}\ifx\mlist\empty\se%
	\errhelp={Tenes que proveer al menos un \integrante{nombre}{lu}{email}.}\fi
	\expandafter\ifsetuperror}


% ----- Reemplazamos el comando \maketitle de LaTeX con el nuestro ---------

\def\maketitle{%
	\ifsetuperror\errmessage{Faltan datos de la caratula! Ingresar 'h' para mas informacion.}\fi
	\thispagestyle{empty}
	\begin{center}
	\vspace*{\stretch{2}}
	{\LARGE\textbf{\Materia}}\\[1em]
	\ifx\Submateria\relax\else{\Large \Submateria}\\[0.5em]\fi
	\par\vspace{\stretch{1}}
	{\large Departamento de Computaci\'on}\\[0.5em]
	{\large Facultad de Ciencias Exactas y Naturales}\\[0.5em]
	{\large Universidad de Buenos Aires}
	\par\vspace{\stretch{3}}
	{\Large \textbf{\Titulo}}\\[0.8em]
	{\Large \Subtitulo}
	\par\vspace{\stretch{3}}
	\ifx\Grupo\relax\else\textbf{\Grupo}\par\bigskip\fi
	\tablaints
	\end{center}
	\vspace*{\stretch{3}}
	\newpage}





\materia{Ingeniería del Software I}
\submateria{Segundo Cuatrimestre de 2013}
\titulo{Trabajo Práctico 1}

\grupo{Grupo 1}
\integrante{Giordano, Mauro}{125/10}{mauro.foxh@gmail.com}
\integrante{Mancuso, Emiliano}{597/07}{emiliano.mancuso@gmail.com}
\integrante{Mataloni, Alejandro}{706/07}{amataloni@gmail.com}
\integrante{Tastzian, Juan Manuel}{039/10}{jm@tast.com.ar}
\integrante{Tolchinsky, Lucas}{591/07}{lucas.tolchinsky@gmail.com}


\begin{titlepage}
\maketitle
\thispagestyle{empty}
\end{titlepage} 

%%%%%%%%%%%%%%%%%%%%%%%%%%%
%				FIN DE CARÁTULA			%
%%%%%%%%%%%%%%%%%%%%%%%%%%%

\tableofcontents
\clearpage

%%%%%%%%%%%%%%%%%%%%%%%%%%%
%					INTRODUCCION			%
%%%%%%%%%%%%%%%%%%%%%%%%%%%

%\include{introduccion}
%\clearpage

%%%%%%%%%%%%%%%%%%%%%%%%%%%
%					DESARROLLO				%
%%%%%%%%%%%%%%%%%%%%%%%%%%%

‌\section{Introducción}
Con este diseño se busca un sistema de venta de telefonía celular que permita tener un alto grado de confiabilidad tanto para el vendedor como para el cliente, que mantenga interconectadas a las distintas partes de la empresa (marketing - ventas - facturación - stock) en todo momento, que automatice algunos procesos internos de la empresa y que agilice y mejore el rendimiento de ventas.\\
\indent Para presentar esta propuesta se incluyen dos alternativas de Diagramas de contexto, donde se visualizan las interacciones entre los distintos agentes y el sistema con un listado del detalle de la misma. Por otro lado, se muestra un Diagrama de Objetivos el cual propone los distintos hitos y metas que deberán cumplirse para lograr satisfacer el objetivo global del diseño. Para cada rama importante de este árbol de objetivos se presenta un detalle de escenarios en donde se ejemplifican, de manera coloquial, distintos sucesos de la vida real acordes al sistema implementado.\\

\section{Modelo Conceptual}

\clearpage

\section{Casos de Uso}

\subsection{Casos de Uso MVP}

\subsubsection{Descripción de Casos de Uso}

\begin{tabular}{ | p{7cm} | p{7cm} | }
  \hline
  \multicolumn{2} {|l|} {Caso de Uso: Autenticándose.} \\
  \multicolumn{2} {|l|} {Actores: Vendedor, Encargado de Stock, Facturación, Marketing.} \\
  \multicolumn{2} {|l|} {Post: El usuario está autenticado en el sistema.} \\
  \hline
  Curso Normal & Curso Alternativo\\
  \hline
  1. El usuario ingresa a la interfaz web del sistema. & \\
  2. El usuario ingresa sus credenciales. & \\
  3. El sistema comprueba la validez de las credenciales y las acepta. & 3. El sistema comprueba la validez de las credenciales y las rechaza\\
  4. El usuario queda autenticado en el sistema. & 4. El usuario es redirigido al formulario para reingresar sus credenciales.\\
  \hline
\end{tabular}

\vspace{1cm}

\begin{tabular}{ | p{14cm} | }
  \hline
  Caso de Uso: Actualizando datos. \\
  Actores: Vendedor. \\
  Pre: El usuario está autenticado en el sistema. \\
  Post: El vendedor tiene actualizados en su dispositivo móvil promociones y datos de clientes. \\
  \hline
  Curso Normal\\
  \hline
  Online o manual (sin auth)?\\
  \hline
\end{tabular}

\vspace{1cm}

\begin{tabular}{ | p{14cm} | }
  \hline
  Caso de Uso: Avisando stock bajo. \\
  Actores: Encargado de Stock. \\
  Pre: El usuario está autenticado en el sistema y l stock de alguno de los equipos asociado a promociones vigentes está pronto a agotarse. \\
  Post: El encargado de stock es notificado sobre el equipo pronto a agotarse. \\
  \hline
  Curso Normal\\
  \hline
  1. El sistema marca como reservados los equipos correspondientes con la promoción vendida.\\
  2. El sistema notifica al encargado de stock sobre los equipos reservados.\\
  \hline
\end{tabular}

\vspace{1cm}

\begin{tabular}{ | p{7cm} | p{7cm} | }
  \hline
  \multicolumn{2} {|l|} {Caso de Uso: Verificando venta.} \\
  \multicolumn{2} {|l|} {Actores: Facturación.} \\
  \multicolumn{2} {|l|} {Pre: El usuario está autenticado con el sistema y se registró una venta con stock disponible.} \\
  \multicolumn{2} {|l|} {Post: La venta se confirma y pasa al sistema de facturación.} \\
  \hline
  Curso Normal & Curso Alternativo\\
  \hline
  1. Facturación verifica la situación del cliente y aprueba la venta. & 1. Facturación verifica la situación del cliente y rechaza la venta. \\
  2. Se pasan los datos al sistema de facturación & 2. Se notifica al vendedor que la venta fue rechazada. \\
  \hline
\end{tabular}

\vspace{1cm}

\begin{tabular}{ | p{7cm} | p{7cm} | }
  \hline
   \multicolumn{2} {| p{10cm} |} {Caso de Uso: Registrando venta.} \\
   \multicolumn{2} {| p{10cm} |} {Actores: Vendedor, Encargado de Stock, Facturación.} \\
   \multicolumn{2} {| p{10cm} |} {Pre: El usuario está autenticado con el sistema y el vendedor tiene datos de clientes y promociones actualizados.} \\
   \multicolumn{2} {| p{10cm} |} {Post: Se registra una venta en el sistema.} \\
  \hline
  Curso Normal & Curso Alternativo\\
  \hline
  1. El vendedor decide junto al cliente qué promoción adquirir. & \\
  2. El vendedor ingresa elige en su dispositivo móvil la promoción que su cliente desea. & \\
  3. El sistema verifica que haya suficientes equipos en stock para satisfacer la promoción. & 3. El sistema verifica que los equipos en stock no alcanzan para satisfacer la promoción\\
  4. El sistema marca como reservados los equipos correspondientes con la promoción vendida. & 4. Se notifica al vendedor que la promoción no puede ser vendida. \\
  5. El sistema notifica al encargado de stock sobre los equipos reservados. & \\ 
  6. Se extiende con Avisando stock bajo. & \\
  7. Usa Verificando venta. & \\
  \hline
\end{tabular}

\vspace{1cm}

\begin{tabular}{ | p{14cm} | }
  \hline
  Caso de Uso: Ingresando equipos nuevos. \\
  Actores: Encargado de Stock. \\
  Pre: El usuario está autenticado con el sistema y nuevos equipos fueron ingresados al almacén. \\
  Post: Los nuevos equipos quedan dados de alta en el sistema. \\
  \hline
  Curso Normal\\
  \hline
  1. El usuario actualiza la cantidad de equipos correspondientes a los ingresados recientemente. \\
  \hline
\end{tabular}

\vspace{1cm}

\begin{tabular}{ | p{14cm} | }
  \hline
  Caso de Uso: Administrando promociones. \\
  Actores: Marketing. \\
  Pre: El usuario está autenticado con el sistema. \\
  Post: Se actualizan las promociones disponibles. \\
  \hline
  Curso Normal\\
  \hline
  1. El departamento de Marketing realiza un estudio de mercado para decidir qué promociones ofrecer según tipo de cliente. \\
  2. El usuario ingresado en el sistema crea, edita o borra una promoción. \\
  \hline
\end{tabular}

\subsection{Casos de Uso}

\subsubsection{Descripción de Casos de Uso}

\clearpage

\section{Diagramas de actividad}

\clearpage

\clearpage

%%%%%%%%%%%%%%%%%%%%%%%%%%%
%					RESULTADOS				%
%%%%%%%%%%%%%%%%%%%%%%%%%%%

\section{Resultados}
\clearpage

%%%%%%%%%%%%%%%%%%%%%%%%%%%
%					ANALISIS				%
%%%%%%%%%%%%%%%%%%%%%%%%%%%

\section{Análisis}
\clearpage

%%%%%%%%%%%%%%%%%%%%%%%%%%%
%					CONCLUSION				%
%%%%%%%%%%%%%%%%%%%%%%%%%%%

\section{Conclusión}
\clearpage

\end{document}
